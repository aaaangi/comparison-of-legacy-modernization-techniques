\chapter{Einleitung}
Legacy-Systeme treten in heutigen Firmen immer öfter auf. Häufig kommt es vor, dass solche Systeme nicht mehr den heutigen Anforderungen bezüglich Technik, Sicherheit, Compliance etc. genügen. Diese Systeme müssen  modernisiert werden, da Wartbarkeit, Effizienz oder Sicherheit nicht mehr gegeben sind und sie das Unternehmen somit finanziellen oder technischen Risiken aussetzen. Mit Hilfe von Modernisierungstechniken können diese Probleme und Risiken behoben werden. Doch welche Modernisierungstechniken entsprechen noch dem heutigen weltweiten Stand der Technik und was für Unterschiede gibt es zwischen den einzelnen Techniken? In dieser Bachelorthesis werden genau solche Modernisierungstechniken untersucht. Dabei werden die Aspekte User Interface, Datenhaltung und Logik/Funktionsumfang analysiert. Weiter wird ein Verfahren entwickelt, um zu entscheiden, ob ein Legacy-System weiterhin gepflegt, modernisiert oder komplett durch ein neues System ersetzt werden soll.

\section{Allgemeine Thematik}
Bei der Suche nach wissenschaftlichen Artikeln ist aufgefallen, dass das Thema «Legacy Systems» selbst schon als «Legacy» bezeichnet werden kann. Es gibt Zahlreiche Dokumentation, die die Problemstellung von Legacy-Systemen aufzeigen und Lösungsansätze anbieten. Die Strategie für Modernisierungen aus dem Paper «An effective strategy for legacy systems evolution» aus dem Jahre 2003 zeigt jedoch auf, dass auch die damaligen Verfahren für die heutige Zeit anwendbar sein können [9].

\section{Definition Legacy-System}
Legacy-Systeme sind Software-Systeme, die in den letzten zwei Jahrzehnten oder länger entwickelt wurden und oft geschäftskritische Informationen enthalten können [9]. Sie laufen meist auf alten Software- und Hardwaretechnologien und -umgebungen. Sie sind schwierig zu modifizieren, teuer zu warten und herausfordernd in neue Technologien und Umgebungen zu integrieren.

\section{Definition Legacy-Modernisierung}
Als Legacy-Modernisierung wird die Transformation eines Legacy-Systems zu einer modernen Anwendung definiert, welche dem heutigen weltweiten Stand der Technik genügt. 

\section{Forschungsfragen}
\label{sec:ff}
In dieser Literatur- und Engineering-Arbeit werden folgende Forschungsfragen untersucht und mit Hilfe eines Proof of Concept an einem bestehenden Legacy-System (Rechteverwaltung) getestet:
\begin{itemize}
	\item \textbf{Forschungsfrage 1:}
\emph{Wie unterscheiden sich die Ansätze der drei Modernisierungstechniken «modellgetriebene Modernisierung», «architekturgetriebene Modernisierung» und «musterbasierte Prozessmodernisierung» in Bezug auf User Interface, Datenhaltung und Logik/Funktionsumfang? }
	\item \textbf{Forschungsfrage 2:}
\emph{Anhand welcher Voraussetzungen kann definiert werden, ob ein Legacy-System modernisiert werden sollte, damit es nicht mehr als Legacy-System gilt und dem aktuellen weltweiten Stand der Technik entspricht?}
\end{itemize}

\section{Abgrenzung und Vorgehensweise}
Ziel der Thesis ist es, drei verschiedene Modernisierungstechniken zu untersuchen und ein Verfahren zu entwickeln, um zu entscheiden, ob ein Legacy-System weiterhin gepflegt, modernisiert oder komplett durch ein neues System ersetzt werden soll. Das zu erarbeitende Verfahren wird im Rahmen eines Proof of Concept an einem bestehenden Legacy-System (Rechteverwaltung) getestet. Das System wurde exemplarisch ausgewählt, da davon ausgegangen wird, dass eine Modernisierung dieses Systems zielführend ist.

\section{Struktur der Arbeit}
Die Struktur der Arbeit ist wie folgt aufgeteilt: In Kapitel 2, 3 und 4 werden die drei Modernisierungstechniken modellgetriebene Modernisierung, architekturgetriebene Modernisierung und musterbasierte Prozessmodernisierung untersucht und zusammengefasst. Durch die Ergebnisse der Kapiteln 2, 3 und 4 wird anschliessend im Kapitel 5 ein Vergleich der drei Modernisierungstechniken aufgezeigt. In Kapitel 6 werden die Voraussetzungen für eine Modernisierung definiert. In Kapitel 7 wird das durchgeführte Resultat des Proof of Concepts beschrieben. Zum Schluss folgt in Kapitel 8 die Diskussion der Arbeit.